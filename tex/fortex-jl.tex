

\documentclass{article}
\usepackage[utf8]{inputenc}
\usepackage{setspace}
\usepackage{ mathrsfs }
\usepackage{graphicx}
\usepackage{amssymb} %maths
\usepackage{amsmath} %maths
\usepackage[margin=0.2in]{geometry}
\usepackage{graphicx}
\usepackage{ulem}
\setlength{\parindent}{0pt}
\setlength{\parskip}{10pt}
\usepackage{hyperref}
\usepackage[autostyle]{csquotes}

\usepackage{cancel}
\renewcommand{\i}{\textit}
\renewcommand{\b}{\textbf}
\newcommand{\q}{\enquote}
%\vskip1.0in





\begin{document}

{\setstretch{0.0}{


\b{Fortex} is a \q{stack of wheels}. It is an intensification of \b{Jagged}, so it's better to understand that program first. 


The essence of the program is in these three functions:

\begin{verbatim}

function stack(f)
    s = 0
    for i in eachindex(f) s += f[i][begin] end
    s
end

function twist!(f,a,n)
    for i in eachindex(f) f[i][1] = mod(f[i][1] + a,n) end
end

function roll!(f, a)
    for i in eachindex(f) f[i] = circshift(f[i],f[mod1(i-1,length(f))][1] + a + 1) end
end


function encode(p,f,n)
    c = Int64[]
    for i in eachindex(p)
        push!(c, mod( stack(f) + p[i], n))
        twist!(f,c[i],n)
        roll!(f, p[i])
    end
    c
end

\end{verbatim}


The big change is that rolling/spinning speed varies between \q{rows} and is a function not only of the latest plaintext symbol but also of  the row above (in a wraparound fashion.) So \b{Fortex} should be more secure than \b{Jagged}. 

\end{document}
