

\documentclass{article}
\usepackage[utf8]{inputenc}
\usepackage{setspace}
\usepackage{ mathrsfs }
\usepackage{graphicx}
\usepackage{amssymb} %maths
\usepackage{amsmath} %maths
\usepackage[margin=0.2in]{geometry}
\usepackage{graphicx}
\usepackage{ulem}
\setlength{\parindent}{0pt}
\setlength{\parskip}{10pt}
\usepackage{hyperref}
\usepackage[autostyle]{csquotes}

\usepackage{cancel}
\renewcommand{\i}{\textit}
\renewcommand{\b}{\textbf}
\newcommand{\q}{\enquote}
%\vskip1.0in





\begin{document}

{\setstretch{0.0}{


\b{Fortex} is a conical or triangular \q{stack of wheels}. Each row is a wheel that spins either clockwise or counterclockwise. A center column functions both for the calculation of a modular sum and as an escalator that increases the number of possible states the machine can be in. 

Earlier versions of Fortex did not use a triangular shape, and this shape is not necessary. Mathematically all that's needed is arrays that are each of a different length and a perpendicular rotation that gets symbols from one row to another. For instance, one can use the first element of each \q{row} as part of such a virtual column. 
 



\end{document}
